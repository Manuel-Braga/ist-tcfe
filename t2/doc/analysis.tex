\section{Theoretical Analysis}
\label{sec:analysis}

In this section, the circuit shown in Figure~\ref{fig:rc} is analysed
theoretically. Using the nodal method, we determine the initial conditions needed. We then proceed to determine the total solution and the frequency response of the circuit.

The data used for the calculations was provided by the $t2\_datagen.py$ file, by inputing the lowest student number of the group members (95780). Those values are presented in the table below:

\begin{table}[h]
  \centering
  \begin{tabular}{|l|r|}
    \hline
    {\bf Name} & {\bf Value} \\ \hline
    \input{../mat/data_TAB}
  \end{tabular}
  \caption{Circuit data generated by inputing 95780 to the $t2\_datagen.py$ file.}                                                            
  \label{tab:data}                                                      
\end{table}   


\subsection{Nodal Method Analysis}

At this point, we are analysing the circuit for $t<0~s$, which is the same as assuming that an infinite amount of time has passed and the capacitator is fully charged, so no current flows through it. This means that we can substitute the capacitor with an open circuit. By doing this, the circuit can be described using only linear components, and so we can use the nodal method to determine a system of linear equations, that can be solved using Octave. These equations are as follows: 



\begin{equation}
\begin{pmatrix}
        -1 & 0 & 0 & 0 & 0 & 0 & 0\\
        G1 & -G1-G2-G3 & G2 & G3 & 0 & 0 & 0\\
        0 & -Kb-G2 & G2 & Kb & 0 & 0 & 0\\
        G1 & -G1 & 0 & -G4 & 0 & -G6 & 0\\
        0 & 0 & 0 & 0 & 0 & G6+G7 & -G7\\
        0 & 0 & 0 & -1 & 0 & -G6*Kd & 1\\
        0 & G3 & 0 & -G3-G4-G5 & G5 & -G6 & 0\\
\end{pmatrix}
\begin{pmatrix}
V1\\
V2\\
V3\\
V5\\
V6\\
V7\\
V8\\
\end{pmatrix}
=
\begin{pmatrix}
-Vs\\
0\\
0\\
0\\
0\\
0\\
0\\
\end{pmatrix}
\end{equation}

\newpage
After this, the current values in each branch are determined using Ohm's Law and are presented, along with the system's solution, in the table below:


\begin{table}[h]
  \centering
  \begin{tabular}{|l|r|}
    \hline
    {\bf Name} & {\bf Value} \\ \hline
    \input{../mat/theorcir1_TAB}
  \end{tabular}
  \caption{Nodal Analysis for $t<0~s$. Currents are expressed in milliAmpere and Voltages are expressed in Volt.}                                                            
  \label{tab:theoretical1}                                                      
\end{table}   


\subsection{Equivalent Resistance and Initial Conditions}

To determine the natural response of the circuit, its time constant ($\tau$) must be known. To discover this, we can determine the Equivalent Resistor $R_{Eq}$ as seen from the capacitor terminals. To do this, $V_{S}$ is set to $0~V$ and the capacitor is replaced by a voltage source $V_{x} = V(6) - V(8)$, where $V(6)$ and $V(8)$ are the voltages in nodes 6 and 8, respectively, as obtained in the previous nodal analysis and shown in Table~\ref{tab:theoretical1}. This is done to ensure that the voltage drop in the capacitor is continuous in time, as in reality, any discontinuity would require an infinite amount of current. 

After this, another nodal analysis is made and another system of equations is determined and solved using Octave. These equations are:

\begin{equation}
\begin{pmatrix}
-1 & 0 & 0 & 0 & 0 & 0 & 0\\
G1 & -G1-G2-G3 & G2 & G3 & 0 & 0 & 0\\
0 & -Kb-G2 & G2 & Kb & 0 & 0 & 0\\
G1 & -G1 & 0 & -G4 & 0 & -G6 & 0\\
0 & 0 & 0 & 0 & 0 & G6+G7 & -G7\\
0 & 0 & 0 & -1 & 0 & -G6*Kd & 1\\
0 & 0 & 0 & 0 & -1 & 0 & 1\\
\end{pmatrix}
\begin{pmatrix}
V1\\
V2\\
V3\\
V5\\
V6\\
V7\\
V8\\
\end{pmatrix}
=
\begin{pmatrix}
0\\
0\\
0\\
0\\
0\\
0\\
-Vx\\
\end{pmatrix}
\end{equation}
\newpage
Other important parameters, such as $I_x$ (the current that is supplied by $V_x$), are defined by the following equations:

\begin{equation}
V_x=V_6-V_8;
\end{equation}
\begin{equation}
I_x=\frac {V_6-V_5}{R_5}+\frac {V_3-V_2}{R_2};
\end{equation}
\begin{equation}
	R_{Eq}=\frac {V_x}{I_x}
\end{equation}
\begin{equation}
	\tau=R_{Eq}\times C;
\end{equation}


After the calculations are done, the values are determined, as presented in the table below:

\begin{table}[h]
  \centering
  \begin{tabular}{|l|r|}
    \hline    
    {\bf Name} & {\bf Value} \\ \hline
    \input{../mat/theorcir2_TAB} 
  \end{tabular}
  \caption{Second nodal analysis and calculations for $R_{Eq}$ and $\tau$. Currents are expressed in $milliAmpere$, Voltages are expressed in $Volt$, Resistances are expressed in $kiloOhm$ and Time Constant is expressed in $second$.}
  \label{tab:theoretical2}
\end{table}

\subsection{Natural Solution}


Using $V_x$ as the initial condition, the natural solution ($V_{6n}$) for this $RC$ circuit is given by the following equation:

\newcommand{\euler}{e}

\begin{equation}
	V_{6n}(t)=V_x e^{(-\frac{t}{\tau})};
\end{equation}
Using Octave, we can plot the results for the interval $[0, 20]~ms$. That plot is shown in the figure below:


\begin{figure}[h] \centering
\includegraphics[width=0.6\linewidth]{natural.eps}
	\caption{Natural solution $V_{6n}$ in the interval $[0, 20]~ms$ plot.}
\label{fig:natural}
\end{figure}


\subsection{Forced Solution}

To be able to analyse the forced response of the circuit, a nodal analysis is made to determine the phasor voltages in all nodes. This is done by using a phasor voltage source $V_S$ of 1 (as shown in Figure~\ref{fig:rc}) and replacing C with its impedance $Z_C$, which leads to the following system of equations, that can once again be solved using Octave:

\begin{equation}
\begin{pmatrix}
1 & 0 & 0 & 0 & 0 & 0 & 0\\
-G1 & G1+G2+G3 & -G2 & -G3 & 0 & 0 & 0\\
0 & Kb+G2 & -G2 & -Kb & 0 & 0 & 0\\
-G1 & G1 & 0 & G4 & 0 & G6 & 0\\
0 & 0 & 0 & 0 & 0 & -G6-G7 & G7\\
0 & 0 & 0 & 1 & 0 & G6*Kd & -1\\
0 & -G3 & 0 & G3+G4+G5 & -G5-jwC & G6 & jwC\\
\end{pmatrix}
\begin{pmatrix}
V1\\
V2\\
V3\\
V5\\
V6\\
V7\\
V8\\
\end{pmatrix}
=
\begin{pmatrix}
-j\\
0\\
0\\
0\\
0\\
0\\
0\\
\end{pmatrix}
\end{equation}


The solution to this system is presented in the table below:


\begin{table}[h]
  \centering
  \begin{tabular}{|l|r|r|}
    \hline    
    {\bf Name} & {\bf Complex Amplitude [V]} & {\bf Phase [Degrees]}\\ \hline
    \input{../mat/forced_tab} 
  \end{tabular}
  \caption{Nodal analysis for phasor voltage in forced state.}
  \label{tab:phasor}
\end{table}


These values are needed to determine the forced solution $V_{6f}$, which is given by the following formula:

\begin{equation}
V_{6f}(t)=V_{6r}cos(\omega t+V_{6\phi});
\end{equation}

Where

\begin{equation}
\omega=2\pi f;
\end{equation}


\subsection{Final Total Solution}


The final total solution $V_6(t)$ is achieved by superimposing the natural and forced solutions. 



\begin{equation}
	V_6(t)=V_{6n}(t)+V_{6f}(t);
\end{equation}
\newpage
By converting the phasors to real time functions for $f = 1 kHz$ and superimposing the natural and forced solutions, we can plot both $V_S(t)$ and $V_6(t)$ in the interval $[-5, 20]~ms$, as shown in the figure below:


\begin{figure}[h] \centering
\includegraphics[width=0.7\linewidth]{total.eps}
\caption{Total solution of $V_{6}$ and $V_{s}$ plot.}
	\label{fig:total}
\end{figure}

As expected, both curves shown in Figure~\ref{fig:total} are constant for $t<0~s$. For $t>0~s$, we can see an evident negative exponencial behavior and an induced frequency in $V_{6}$.


\subsection{Frequency Responses}

Because $V_S(t) = sin(\omega t)$, the magnitude and phase are independent of the frequency $f$. This means that both the magnitude and phase are expected to be constant in the plots that follow.

The magnitude frequency response is given in Figure~\ref{fig:mag}.
The phase response for frequencies ranging from $0.1~Hz$ to $1~MHz$ is given in Figure~\ref{fig:phase}. The apparent discontinuity showed in the phase of $V(6)$ is in fact caused by the domain of the arctan function that is used to determined the phase, which is, in reality, continuous.

\begin{figure}[h] \centering
\includegraphics[width=0.5\linewidth]{magnitude.eps}
	\caption{Magnitude frequency response plot, in dB, of $V_{c}$, $V(6)$ and $V_{s}$. Frequencies ranging from $0.1~Hz$ to $1~MHz$ in logarithmic scale.}
        \label{fig:mag}
\end{figure}


\begin{figure}[h] \centering
\includegraphics[width=0.7\linewidth]{phase.eps}
	\caption{Phase response plot, in degrees, of $V_{c}$, $V(6)$ and $V_{s}$. Frequencies ranging from $0.1~Hz$ to $1~MHz$ in logarithmic scale.}
        \label{fig:phase}
\end{figure}



The circuit being analysed can be used as a low-pass filter. This means that, for low frequencies, the capacitator has time to charge up to almost the same voltage provided as input (it approximates to an open-circuit behavior), which translates to a proximity in phase between the voltage in the capacitor and the voltage source. Having said that, high frequencies, on the other hand, will give the capacitor a small time to charge up before a change in the input direction occurs (it approximates to a short-circuit behavior), which translates to a growing difference in phase between the voltage in the capacitor and the voltage source. This difference of phase is noticeable for frequencies greater than the cut-off frequency, $f_c$, which is given by $f_c = \frac{1}{2\pi\tau}$. In this particular case, the cut-off frequency value is approximately $50~Hz$. This is why we see a significant voltage drop in Fig~\ref{fig:mag} at around $10~Hz$ and $10^2~Hz$. The phase difference also starts to increase in that same range, as seen in Fig~\ref{fig:phase}.

By simplifying this circuit to an equivalent one composed by a voltage source, capacitor and equivalent resistor, we reach the following equations that help us to better comprehend the magnitude drop and the phase difference rise with the increase of frequency:

\begin{equation}
  V_c = \frac{V_s}{\sqrt{1 + (R_{Eq}\times C\times 2\pi\times f)^2}}
  \label{eq:equivalent1}
\end{equation}

\begin{equation}
  \phi_{V_c} = -\frac{\pi}{2} + arctan(R_{Eq}\times C\times 2\pi\times f)
  \label{eq:equivalent2}
\end{equation}

