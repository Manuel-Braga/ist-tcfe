\section{Theoretical Analysis}
\label{sec:analysis}

In this section, the circuit shown in Figure~\ref{fig:rc} is analysed
theoretically, in terms of the voltages and current intensities in the various branches and components of the circuit.

\section{______}

The circuit consists of multiple loops with various values of current, $i$ circulating in each of the branches of the circuit. There are two
voltage sources, $v_a$ and $v_c$ driving their inputs. Applying the Mesh Method, we reach a system of 5 equations and unkowns, that we solved in Octave. Those equations can be written as

\begin{equation}
  (v_2 - v_5)G_3 + (v_2 - v_1)G_1 + (v_2 - v_3)G_2 = 0.
  \label{eq:kvl}
\end{equation}

\begin{equation}
  (v_3 - v_2)G_2 - i_b = 0.
\end{equation}

\begin{equation}
  i_b - i_d + (v_4 - v_5)G_5 = 0.
  \label{eq:kvl2}
\end{equation}

$i_b = K_b$

Equation~(\ref{eq:kvl2}) is a linear differencial equation whose solution is a
superposition of a natural solution $v_{On}$ and a forced solution $v_{Of}$:

\begin{equation}
  v_O(t) = v_{On}(t) + v_{Of}(t).
  \label{eq:vo_sol}
\end{equation}

As learned in the theory classes the natural solution is of the form
\begin{equation}
  v_{On}(t) = Ae^{-\frac{t}{RC}},
  \label{eq:vo_nat}
\end{equation}
where $A$ is an integration constant.

The forced solution is of the form given in Equation~(\ref{eq:vo_for}) and is
illustrated in Figure~\ref{fig:forced}.

\begin{equation}
  V_{Of}(t) = |\bar{V}_{Of}| cos(\omega t + \angle \bar{V}_{Of}),
  \label{eq:vo_for}
\end{equation}

\begin{figure}[h] \centering
\includegraphics[width=0.8\linewidth]{forced.eps}
\caption{------------}
\label{fig:forced}
\end{figure}

\section{-------}

