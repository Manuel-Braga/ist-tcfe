\section{Simulation Analysis}
\label{sec:simulation}

\subsection{Operating Point Analysis}

In order to properly analyse this circuit using Ngspice, it was convenient to slightly modify it. By defining the Ground voltage reference, we added a 0V voltage source between nodes 0 and 1, as shown in PÔR FIGURA. In order to correctly define Vc, which is a CCVS that therefore requires a voltage source through which the controlling current flows, we had to add to the circuit another 0V voltage source. Because we knew that current Ic (which controlled Vc) flew through R6, we decided to add the aforementioned voltage source between resistors R6 and R7.
Despite being necessary to the sucess of the simulation, these modifications create a totally equivalent circuit. This means that these additions do not compromise the analysis of the original circuit in any way.


Table~\ref{tab:op} shows the simulated operating point results for the circuit
under analysis. Compared to the theoretical analysis results, one notices the
following differences: describe and explain the differences.

\begin{table}[h]
  \centering
  \begin{tabular}{|l|r|}
    \hline    
    {\bf Name} & {\bf Value [A or V]} \\ \hline
    @cb[i] & 0.000000e+00\\ \hline
@ce[i] & 0.000000e+00\\ \hline
@q1[ib] & 7.022567e-05\\ \hline
@q1[ic] & 1.404513e-02\\ \hline
@q1[ie] & -1.41154e-02\\ \hline
@q1[is] & 5.765392e-12\\ \hline
@rc[i] & 1.411536e-02\\ \hline
@re[i] & 1.411536e-02\\ \hline
@rf[i] & 7.022567e-05\\ \hline
@rs[i] & 0.000000e+00\\ \hline
v(1) & 0.000000e+00\\ \hline
v(2) & 0.000000e+00\\ \hline
base & 2.254108e+00\\ \hline
coll & 5.765392e+00\\ \hline
emit & 1.411536e+00\\ \hline
vcc & 1.000000e+01\\ \hline

  \end{tabular}
  \caption{Operating point. A variable preceded by @ is of type {\em current}
    and expressed in Ampere; other variables are of type {\it voltage} and expressed in
    Volt.}
  \label{tab:op}
\end{table}



